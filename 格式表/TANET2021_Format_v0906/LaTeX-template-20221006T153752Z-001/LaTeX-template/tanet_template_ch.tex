\documentclass[conference]{IEEEtran}
\IEEEoverridecommandlockouts
% The preceding line is only needed to identify funding in the first footnote. If that is unneeded, please comment it out.
\usepackage{cite}
\usepackage{amsmath,amssymb,amsfonts}
\usepackage{algorithmic}
\usepackage{graphicx}
\usepackage{textcomp}
\usepackage{xcolor}

\usepackage{fontspec}
\usepackage{xeCJK}
\renewcommand\tablename{表}
\renewcommand\figurename{圖}

\setmainfont{Times New Roman}
\setCJKmainfont{標楷體} % 設定中文字型

\def\BibTeX{{\rm B\kern-.05em{\sc i\kern-.025em b}\kern-.08em
    T\kern-.1667em\lower.7ex\hbox{E}\kern-.125emX}}
\columnsep 0.3cm
\begin{document}

\title{TANET2022研討會論文格式說明
\thanks{補助單位}
}

\author{\IEEEauthorblockN{1\textsuperscript{st} 作者一}
\IEEEauthorblockA{\textit{部門名稱} \\
\textit{服務單位}\\
Email}
\and
\IEEEauthorblockN{2\textsuperscript{nd} 作者二}
\IEEEauthorblockA{\textit{部門名稱} \\
\textit{服務單位}\\
Email}
\and
\IEEEauthorblockN{3\textsuperscript{rd} 作者三}
\IEEEauthorblockA{\textit{部門名稱} \\
\textit{服務單位}\\
Email}
\and
\IEEEauthorblockN{4\textsuperscript{th} 作者四}
\IEEEauthorblockA{\textit{部門名稱} \\
\textit{服務單位}\\
Email}
\and
\IEEEauthorblockN{5\textsuperscript{th} 作者五}
\IEEEauthorblockA{\textit{部門名稱} \\
\textit{服務單位}\\
Email}
\and
\IEEEauthorblockN{6\textsuperscript{th} 作者六}
\IEEEauthorblockA{\textit{部門名稱} \\
\textit{服務單位}\\
Email}
}

\maketitle


\begin{center}
摘要
\end{center}
本文將說明TANET2022研討會的定稿排版格式,已由本研討會評審接受的論文,煩請務必依照本格式進行編排。若未依規定進行排版者,將不予列入審查。
本文檔是 {\tt XeLaTeX} 的模型和說明。
該文件和 IEEEtran.cls 文件定義了論文的組成部分 [標題、文本、標題等]。 *CRITICAL:論文標題或摘要中請勿使用符號、特殊字符、腳註、
或數學公式。

\noindent {\bf 關鍵字:} 元件, 格式, 風格



\section{前言}
論文請用A4紙依本格式撰寫並精簡至6頁,為印刷品質起見,建議以解析度為2400dpi或高品質以上之pdf上傳。

\section{易用性}

\subsection{稿件規範}

本文使用IEEEtran class 定義整體格式和設置文本樣式包含頁邊空白、行距、列寬、行距和文本字體;請不要修改任何預設頁邊空白、行距、列寬、行距和文本字體。

\section{先完成內容再開始設定格式}
在開始設定您的論文格式之前,首先將論文內容寫入並保存為單獨的文件。在完成所有內容編輯後再開始進行論文的
格式設定。請參考下面的 \ref{AA}--\ref{SCM} 部分以獲取更多關於校對、拼寫和語法的相關資訊。

將文本和圖形文件分開,直到文本完成格式化和樣式化。不要給文本標題編號---{\LaTeX} 會這樣做
為你。

\subsection{縮寫與首字母縮略詞}\label{AA}
第一次在文本中使用時定義縮寫詞和首字母縮略詞,
即使它們已經在摘要中定義。縮寫如
不必定義 IEEE、SI、MKS、CGS、ac、dc 和 rms。
標題或標題中不使用縮寫,除非它們是不可避免的。




\subsection{Equations}
論文中之數學方程式以阿拉伯數字逐一按出現或引用順序編碼,並加小括號“()”表示之,例如,第一個方程式應表示成“(1)”。展列(display)之方程式應置於版面中間,各方程式編碼一律置於每式在欄之最右側切齊。如下面之(1)式數學公式/方程式:
\begin{equation}
a+b=\gamma\label{eq}
\end{equation}


確保
等式中的符號已在之前或之後定義
方程。使用 ``\eqref{eq}'',而不是 ``公式~\eqref{eq}'' 或 ``公式 \eqref{eq}'',除非在
句首:``公式 \eqref{eq} 是 . . ''




\subsection{作者和服務單位}
文章應至少列出一位作者。作者姓名
應該從左到右列出,然後向下移動到
下一行。這是將在未來引用中使用的作者序列
並通過索引服務。名稱不應列在列中或分組依據
聯繫。請保持您的從屬關係盡可能簡潔(對於
例如,不要區分同一組織的部門)。

\subsection{列出標題}
標題訂定可以引導讀者可以快速了解論文的組織結構。
標題訂定分為元件標題和文本標題兩類。

元件標題是為論文中不同的元件組成設定標題,這些元件彼此沒有
從屬關係。例如誌謝辭和
參考文獻。

文本標題在相關的、層次結構的基礎上組織主題。
例如,論文標題是主要文本標題,因為所有後續
材料涉及並詳細闡述了這一主題。如果有兩個或更多
子主題,應使用下一級標題(大寫羅馬數字)
反之,如果沒有至少兩個子主題,則沒有子標題
應該介紹。

\subsection{圖片和表格}
將圖形和表格放在左欄或右欄的頂部或置於底部。避免將它們放在左欄或右欄的中間。大的
圖形和表格可能跨越兩欄。圖標題應該是在數字下方;表說明文字應出現在表格上方。插入
文中引用後的圖形和表格。使用
``圖~\ref{fig}'',表~\ref{tab1}。

\begin{table}[htbp]
\caption{表格範例}
\begin{center}
\begin{tabular}{|c|c|c|c|}
\hline
\textbf{Table}&\multicolumn{3}{|c|}{\textbf{表格列頭}} \\
\cline{2-4}
\textbf{標題} & \textbf{\textit{表格列副標題}}& \textbf{\textit{副標題}}& \textbf{\textit{副標題}} \\
\hline
列標題& 表格內容$^{\mathrm{a}}$& &  \\
\hline
\multicolumn{4}{l}{$^{\mathrm{a}}$表格內容腳註}
\end{tabular}
\label{tab1}
\end{center}
\end{table}

\begin{figure}[htbp]
\centerline{\includegraphics{fig1.png}}
\caption{圖範例}
\label{fig}
\end{figure}



\section*{誌謝辭}

感謝***在本研究進行時,提供...協助。

論文研究成果若為計畫補助者,請列在第一頁的未編號腳註中。

\section*{參考文獻}

請在括號 \cite{b1} 內連續編號引用。這
句子標點在括號 \cite{b2} 之後。簡單參考參考
數字,如 \cite{b3} --- 不要使用 ``Ref. \cite{b3}'' 或 ``reference \cite{b3}'' 除了在
句首:``Reference \cite{b3} is the first $\ldots$''

在上標中單獨編號腳註。將實際腳註放在
引用它的列的底部。不要在正文中添加腳註
摘要或參考列表。使用字母作為表格腳註。

除非有六位或更多作者,否則請給出所有作者的姓名;不使用
``等''。尚未發表的論文,即使已發表
提交出版,應引用為“未出版”\cite{b4}。文件
已被接受出版的應該被引用為“in press”\cite{b5}。
論文標題中的第一個字字首大寫,專有名詞和
元素符號。

發表於中文期刊的論文,請提供中文文獻。
發表於外文期刊的論文,請提供外文文獻。



\begin{thebibliography}{00}
\bibitem{b1} G. Eason, B. Noble, and I. N. Sneddon, ``On certain integrals of Lipschitz-Hankel type involving products of Bessel functions,'' Phil. Trans. Roy. Soc. London, vol. A247, pp. 529--551, April 1955.
\bibitem{b2} J. Clerk Maxwell, A Treatise on Electricity and Magnetism, 3rd ed., vol. 2. Oxford: Clarendon, 1892, pp.68--73.
\bibitem{b3} I. S. Jacobs and C. P. Bean, ``Fine particles, thin films and exchange anisotropy,'' in Magnetism, vol. III, G. T. Rado and H. Suhl, Eds. New York: Academic, 1963, pp. 271--350.
\bibitem{b4} K. Elissa, ``Title of paper if known,'' unpublished.
\bibitem{b5} R. Nicole, ``Title of paper with only first word capitalized,'' J. Name Stand. Abbrev., in press.
\bibitem{b6} Y. Yorozu, M. Hirano, K. Oka, and Y. Tagawa, ``Electron spectroscopy studies on magneto-optical media and plastic substrate interface,'' IEEE Transl. J. Magn. Japan, vol. 2, pp. 740--741, August 1987 [Digests 9th Annual Conf. Magnetics Japan, p. 301, 1982].
\bibitem{b7} M. Young, The Technical Writer's Handbook. Mill Valley, CA: University Science, 1989.
\end{thebibliography}


\end{document}
